
% Default to the notebook output style

    


% Inherit from the specified cell style.




    
\documentclass{article}

    
    
    \usepackage{graphicx} % Used to insert images
    \usepackage{adjustbox} % Used to constrain images to a maximum size 
    \usepackage{color} % Allow colors to be defined
    \usepackage{enumerate} % Needed for markdown enumerations to work
    \usepackage{geometry} % Used to adjust the document margins
    \usepackage{amsmath} % Equations
    \usepackage{amssymb} % Equations
    \usepackage[mathletters]{ucs} % Extended unicode (utf-8) support
    \usepackage[utf8x]{inputenc} % Allow utf-8 characters in the tex document
    \usepackage{fancyvrb} % verbatim replacement that allows latex
    \usepackage{grffile} % extends the file name processing of package graphics 
                         % to support a larger range 
    % The hyperref package gives us a pdf with properly built
    % internal navigation ('pdf bookmarks' for the table of contents,
    % internal cross-reference links, web links for URLs, etc.)
    \usepackage{hyperref}
    \usepackage{longtable} % longtable support required by pandoc >1.10
    \usepackage{booktabs}  % table support for pandoc > 1.12.2
    

    
    
    \definecolor{orange}{cmyk}{0,0.4,0.8,0.2}
    \definecolor{darkorange}{rgb}{.71,0.21,0.01}
    \definecolor{darkgreen}{rgb}{.12,.54,.11}
    \definecolor{myteal}{rgb}{.26, .44, .56}
    \definecolor{gray}{gray}{0.45}
    \definecolor{lightgray}{gray}{.95}
    \definecolor{mediumgray}{gray}{.8}
    \definecolor{inputbackground}{rgb}{.95, .95, .85}
    \definecolor{outputbackground}{rgb}{.95, .95, .95}
    \definecolor{traceback}{rgb}{1, .95, .95}
    % ansi colors
    \definecolor{red}{rgb}{.6,0,0}
    \definecolor{green}{rgb}{0,.65,0}
    \definecolor{brown}{rgb}{0.6,0.6,0}
    \definecolor{blue}{rgb}{0,.145,.698}
    \definecolor{purple}{rgb}{.698,.145,.698}
    \definecolor{cyan}{rgb}{0,.698,.698}
    \definecolor{lightgray}{gray}{0.5}
    
    % bright ansi colors
    \definecolor{darkgray}{gray}{0.25}
    \definecolor{lightred}{rgb}{1.0,0.39,0.28}
    \definecolor{lightgreen}{rgb}{0.48,0.99,0.0}
    \definecolor{lightblue}{rgb}{0.53,0.81,0.92}
    \definecolor{lightpurple}{rgb}{0.87,0.63,0.87}
    \definecolor{lightcyan}{rgb}{0.5,1.0,0.83}
    
    % commands and environments needed by pandoc snippets
    % extracted from the output of `pandoc -s`
    \DefineVerbatimEnvironment{Highlighting}{Verbatim}{commandchars=\\\{\}}
    % Add ',fontsize=\small' for more characters per line
    \newenvironment{Shaded}{}{}
    \newcommand{\KeywordTok}[1]{\textcolor[rgb]{0.00,0.44,0.13}{\textbf{{#1}}}}
    \newcommand{\DataTypeTok}[1]{\textcolor[rgb]{0.56,0.13,0.00}{{#1}}}
    \newcommand{\DecValTok}[1]{\textcolor[rgb]{0.25,0.63,0.44}{{#1}}}
    \newcommand{\BaseNTok}[1]{\textcolor[rgb]{0.25,0.63,0.44}{{#1}}}
    \newcommand{\FloatTok}[1]{\textcolor[rgb]{0.25,0.63,0.44}{{#1}}}
    \newcommand{\CharTok}[1]{\textcolor[rgb]{0.25,0.44,0.63}{{#1}}}
    \newcommand{\StringTok}[1]{\textcolor[rgb]{0.25,0.44,0.63}{{#1}}}
    \newcommand{\CommentTok}[1]{\textcolor[rgb]{0.38,0.63,0.69}{\textit{{#1}}}}
    \newcommand{\OtherTok}[1]{\textcolor[rgb]{0.00,0.44,0.13}{{#1}}}
    \newcommand{\AlertTok}[1]{\textcolor[rgb]{1.00,0.00,0.00}{\textbf{{#1}}}}
    \newcommand{\FunctionTok}[1]{\textcolor[rgb]{0.02,0.16,0.49}{{#1}}}
    \newcommand{\RegionMarkerTok}[1]{{#1}}
    \newcommand{\ErrorTok}[1]{\textcolor[rgb]{1.00,0.00,0.00}{\textbf{{#1}}}}
    \newcommand{\NormalTok}[1]{{#1}}
    
    % Define a nice break command that doesn't care if a line doesn't already
    % exist.
    \def\br{\hspace*{\fill} \\* }
    % Math Jax compatability definitions
    \def\gt{>}
    \def\lt{<}
    % Document parameters
    
\title{Minimización por el método de Newton-Raphson}

    
    
\author{Pablo de Castro}

    

    % Pygments definitions
    
\makeatletter
\def\PY@reset{\let\PY@it=\relax \let\PY@bf=\relax%
    \let\PY@ul=\relax \let\PY@tc=\relax%
    \let\PY@bc=\relax \let\PY@ff=\relax}
\def\PY@tok#1{\csname PY@tok@#1\endcsname}
\def\PY@toks#1+{\ifx\relax#1\empty\else%
    \PY@tok{#1}\expandafter\PY@toks\fi}
\def\PY@do#1{\PY@bc{\PY@tc{\PY@ul{%
    \PY@it{\PY@bf{\PY@ff{#1}}}}}}}
\def\PY#1#2{\PY@reset\PY@toks#1+\relax+\PY@do{#2}}

\expandafter\def\csname PY@tok@gd\endcsname{\def\PY@tc##1{\textcolor[rgb]{0.63,0.00,0.00}{##1}}}
\expandafter\def\csname PY@tok@gu\endcsname{\let\PY@bf=\textbf\def\PY@tc##1{\textcolor[rgb]{0.50,0.00,0.50}{##1}}}
\expandafter\def\csname PY@tok@gt\endcsname{\def\PY@tc##1{\textcolor[rgb]{0.00,0.27,0.87}{##1}}}
\expandafter\def\csname PY@tok@gs\endcsname{\let\PY@bf=\textbf}
\expandafter\def\csname PY@tok@gr\endcsname{\def\PY@tc##1{\textcolor[rgb]{1.00,0.00,0.00}{##1}}}
\expandafter\def\csname PY@tok@cm\endcsname{\let\PY@it=\textit\def\PY@tc##1{\textcolor[rgb]{0.25,0.50,0.50}{##1}}}
\expandafter\def\csname PY@tok@vg\endcsname{\def\PY@tc##1{\textcolor[rgb]{0.10,0.09,0.49}{##1}}}
\expandafter\def\csname PY@tok@m\endcsname{\def\PY@tc##1{\textcolor[rgb]{0.40,0.40,0.40}{##1}}}
\expandafter\def\csname PY@tok@mh\endcsname{\def\PY@tc##1{\textcolor[rgb]{0.40,0.40,0.40}{##1}}}
\expandafter\def\csname PY@tok@go\endcsname{\def\PY@tc##1{\textcolor[rgb]{0.53,0.53,0.53}{##1}}}
\expandafter\def\csname PY@tok@ge\endcsname{\let\PY@it=\textit}
\expandafter\def\csname PY@tok@vc\endcsname{\def\PY@tc##1{\textcolor[rgb]{0.10,0.09,0.49}{##1}}}
\expandafter\def\csname PY@tok@il\endcsname{\def\PY@tc##1{\textcolor[rgb]{0.40,0.40,0.40}{##1}}}
\expandafter\def\csname PY@tok@cs\endcsname{\let\PY@it=\textit\def\PY@tc##1{\textcolor[rgb]{0.25,0.50,0.50}{##1}}}
\expandafter\def\csname PY@tok@cp\endcsname{\def\PY@tc##1{\textcolor[rgb]{0.74,0.48,0.00}{##1}}}
\expandafter\def\csname PY@tok@gi\endcsname{\def\PY@tc##1{\textcolor[rgb]{0.00,0.63,0.00}{##1}}}
\expandafter\def\csname PY@tok@gh\endcsname{\let\PY@bf=\textbf\def\PY@tc##1{\textcolor[rgb]{0.00,0.00,0.50}{##1}}}
\expandafter\def\csname PY@tok@ni\endcsname{\let\PY@bf=\textbf\def\PY@tc##1{\textcolor[rgb]{0.60,0.60,0.60}{##1}}}
\expandafter\def\csname PY@tok@nl\endcsname{\def\PY@tc##1{\textcolor[rgb]{0.63,0.63,0.00}{##1}}}
\expandafter\def\csname PY@tok@nn\endcsname{\let\PY@bf=\textbf\def\PY@tc##1{\textcolor[rgb]{0.00,0.00,1.00}{##1}}}
\expandafter\def\csname PY@tok@no\endcsname{\def\PY@tc##1{\textcolor[rgb]{0.53,0.00,0.00}{##1}}}
\expandafter\def\csname PY@tok@na\endcsname{\def\PY@tc##1{\textcolor[rgb]{0.49,0.56,0.16}{##1}}}
\expandafter\def\csname PY@tok@nb\endcsname{\def\PY@tc##1{\textcolor[rgb]{0.00,0.50,0.00}{##1}}}
\expandafter\def\csname PY@tok@nc\endcsname{\let\PY@bf=\textbf\def\PY@tc##1{\textcolor[rgb]{0.00,0.00,1.00}{##1}}}
\expandafter\def\csname PY@tok@nd\endcsname{\def\PY@tc##1{\textcolor[rgb]{0.67,0.13,1.00}{##1}}}
\expandafter\def\csname PY@tok@ne\endcsname{\let\PY@bf=\textbf\def\PY@tc##1{\textcolor[rgb]{0.82,0.25,0.23}{##1}}}
\expandafter\def\csname PY@tok@nf\endcsname{\def\PY@tc##1{\textcolor[rgb]{0.00,0.00,1.00}{##1}}}
\expandafter\def\csname PY@tok@si\endcsname{\let\PY@bf=\textbf\def\PY@tc##1{\textcolor[rgb]{0.73,0.40,0.53}{##1}}}
\expandafter\def\csname PY@tok@s2\endcsname{\def\PY@tc##1{\textcolor[rgb]{0.73,0.13,0.13}{##1}}}
\expandafter\def\csname PY@tok@vi\endcsname{\def\PY@tc##1{\textcolor[rgb]{0.10,0.09,0.49}{##1}}}
\expandafter\def\csname PY@tok@nt\endcsname{\let\PY@bf=\textbf\def\PY@tc##1{\textcolor[rgb]{0.00,0.50,0.00}{##1}}}
\expandafter\def\csname PY@tok@nv\endcsname{\def\PY@tc##1{\textcolor[rgb]{0.10,0.09,0.49}{##1}}}
\expandafter\def\csname PY@tok@s1\endcsname{\def\PY@tc##1{\textcolor[rgb]{0.73,0.13,0.13}{##1}}}
\expandafter\def\csname PY@tok@sh\endcsname{\def\PY@tc##1{\textcolor[rgb]{0.73,0.13,0.13}{##1}}}
\expandafter\def\csname PY@tok@sc\endcsname{\def\PY@tc##1{\textcolor[rgb]{0.73,0.13,0.13}{##1}}}
\expandafter\def\csname PY@tok@sx\endcsname{\def\PY@tc##1{\textcolor[rgb]{0.00,0.50,0.00}{##1}}}
\expandafter\def\csname PY@tok@bp\endcsname{\def\PY@tc##1{\textcolor[rgb]{0.00,0.50,0.00}{##1}}}
\expandafter\def\csname PY@tok@c1\endcsname{\let\PY@it=\textit\def\PY@tc##1{\textcolor[rgb]{0.25,0.50,0.50}{##1}}}
\expandafter\def\csname PY@tok@kc\endcsname{\let\PY@bf=\textbf\def\PY@tc##1{\textcolor[rgb]{0.00,0.50,0.00}{##1}}}
\expandafter\def\csname PY@tok@c\endcsname{\let\PY@it=\textit\def\PY@tc##1{\textcolor[rgb]{0.25,0.50,0.50}{##1}}}
\expandafter\def\csname PY@tok@mf\endcsname{\def\PY@tc##1{\textcolor[rgb]{0.40,0.40,0.40}{##1}}}
\expandafter\def\csname PY@tok@err\endcsname{\def\PY@bc##1{\setlength{\fboxsep}{0pt}\fcolorbox[rgb]{1.00,0.00,0.00}{1,1,1}{\strut ##1}}}
\expandafter\def\csname PY@tok@kd\endcsname{\let\PY@bf=\textbf\def\PY@tc##1{\textcolor[rgb]{0.00,0.50,0.00}{##1}}}
\expandafter\def\csname PY@tok@ss\endcsname{\def\PY@tc##1{\textcolor[rgb]{0.10,0.09,0.49}{##1}}}
\expandafter\def\csname PY@tok@sr\endcsname{\def\PY@tc##1{\textcolor[rgb]{0.73,0.40,0.53}{##1}}}
\expandafter\def\csname PY@tok@mo\endcsname{\def\PY@tc##1{\textcolor[rgb]{0.40,0.40,0.40}{##1}}}
\expandafter\def\csname PY@tok@kn\endcsname{\let\PY@bf=\textbf\def\PY@tc##1{\textcolor[rgb]{0.00,0.50,0.00}{##1}}}
\expandafter\def\csname PY@tok@mi\endcsname{\def\PY@tc##1{\textcolor[rgb]{0.40,0.40,0.40}{##1}}}
\expandafter\def\csname PY@tok@gp\endcsname{\let\PY@bf=\textbf\def\PY@tc##1{\textcolor[rgb]{0.00,0.00,0.50}{##1}}}
\expandafter\def\csname PY@tok@o\endcsname{\def\PY@tc##1{\textcolor[rgb]{0.40,0.40,0.40}{##1}}}
\expandafter\def\csname PY@tok@kr\endcsname{\let\PY@bf=\textbf\def\PY@tc##1{\textcolor[rgb]{0.00,0.50,0.00}{##1}}}
\expandafter\def\csname PY@tok@s\endcsname{\def\PY@tc##1{\textcolor[rgb]{0.73,0.13,0.13}{##1}}}
\expandafter\def\csname PY@tok@kp\endcsname{\def\PY@tc##1{\textcolor[rgb]{0.00,0.50,0.00}{##1}}}
\expandafter\def\csname PY@tok@w\endcsname{\def\PY@tc##1{\textcolor[rgb]{0.73,0.73,0.73}{##1}}}
\expandafter\def\csname PY@tok@kt\endcsname{\def\PY@tc##1{\textcolor[rgb]{0.69,0.00,0.25}{##1}}}
\expandafter\def\csname PY@tok@ow\endcsname{\let\PY@bf=\textbf\def\PY@tc##1{\textcolor[rgb]{0.67,0.13,1.00}{##1}}}
\expandafter\def\csname PY@tok@sb\endcsname{\def\PY@tc##1{\textcolor[rgb]{0.73,0.13,0.13}{##1}}}
\expandafter\def\csname PY@tok@k\endcsname{\let\PY@bf=\textbf\def\PY@tc##1{\textcolor[rgb]{0.00,0.50,0.00}{##1}}}
\expandafter\def\csname PY@tok@se\endcsname{\let\PY@bf=\textbf\def\PY@tc##1{\textcolor[rgb]{0.73,0.40,0.13}{##1}}}
\expandafter\def\csname PY@tok@sd\endcsname{\let\PY@it=\textit\def\PY@tc##1{\textcolor[rgb]{0.73,0.13,0.13}{##1}}}

\def\PYZbs{\char`\\}
\def\PYZus{\char`\_}
\def\PYZob{\char`\{}
\def\PYZcb{\char`\}}
\def\PYZca{\char`\^}
\def\PYZam{\char`\&}
\def\PYZlt{\char`\<}
\def\PYZgt{\char`\>}
\def\PYZsh{\char`\#}
\def\PYZpc{\char`\%}
\def\PYZdl{\char`\$}
\def\PYZhy{\char`\-}
\def\PYZsq{\char`\'}
\def\PYZdq{\char`\"}
\def\PYZti{\char`\~}
% for compatibility with earlier versions
\def\PYZat{@}
\def\PYZlb{[}
\def\PYZrb{]}
\makeatother


    % Exact colors from NB
    \definecolor{incolor}{rgb}{0.0, 0.0, 0.5}
    \definecolor{outcolor}{rgb}{0.545, 0.0, 0.0}



    
    % Prevent overflowing lines due to hard-to-break entities
    \sloppy 
    % Setup hyperref package
    \hypersetup{
      breaklinks=true,  % so long urls are correctly broken across lines
      colorlinks=true,
      urlcolor=blue,
      linkcolor=darkorange,
      citecolor=darkgreen,
      }
    % Slightly bigger margins than the latex defaults
    
    \geometry{verbose,tmargin=1in,bmargin=1in,lmargin=1in,rmargin=1in}
    
    

    \begin{document}
    
    
    \maketitle
    
    

    

    \section{Introducción}


    El objetivo de este trabajo es la implementación de un programa para el
ajuste no lineal por mínimos cuadrados mediante minimización del
\(\chi^2\) por el método de Newton-Raphson. Este programa se va a
utiizar para ajustar unos datos experimentales a una función
\(y=\exp(ax)+bx^2\) para estimar los parámetros \(a\) y \(b\) óptimos.
Adicionalmente, se van a estimar las incertumbres del ajuste a partir de
la matriz de información de Fisher y simulaciones Monte Carlo.


    \section{Fundamento Teórico}


    Sea un conjunto de \(N\) medidas \(y_i\) (con errores \(\sigma_i\)) para
valores \(x_i\) a los que se les quiere ajustar una función
\(f(x_i;a_j)\) que predice los valores de \(y_i\), donde \(a_j\) con
\(j=1,..,p\) son parámetros desconocidos. Los valores óptimos de los
parámetros \(a_j\), de acuerdo con el método de mínimos cuadrados, para
el ajuste se pueden estimar minimizando el \(\chi^2\):

\[\chi^2 = \sum_{i=1}^{N} \frac{[y_i -f(x_i;a_j)]^2}{\sigma_i^2} \]

que sigue aproximadamente una distribución \(\chi^2\) con \(N-p\) grados
de libertad, hecho que puede utilizarse para evaluar la bondad del
ajuste como la probabilidad del \(\chi^2\) real sea mayor que el
\(\chi2\) observado.

La minimización para funciones sencillas se puede realizar de forma
analítica, sin embargo, en general se requiere la minimización numérica
del \(\chi^2\) para la obtención de los parámetros. Existen multitud de
métodos de minimización numérica, aunque el método que se va utilizar en
este trabajo es el Newton-Raphson (NR) que sirve para buscar el cero de
la función, por lo que se puede utilizar para buscar el cero de del
gradiente de \(\chi^2\) con respecto a los parámetros.

Supóngase una función \(f(x_1,...,x_n)\) que depende de varios
parametros \(\bf{x}\), se puede desarrollar en serie en primer orden
alrededor de un valor inicial \(\bf{x^{(0)}}\) según:

\[\nabla f({\bf{x}}) = \nabla f ({\bf{x}}^{(0)}) 
+ \bf{H}({\bf{x}}- {\bf{x}}^{(0)}) ~+~ ... \]

en donde \(\bf{x}\) es el vector de parámetros, \(\nabla f({\bf{x}})\)
es el vector gradiente y \(\bf{H}\) es la matriz hessiana. Si se impone
que \(\nabla f ({\bf{x}}^{(0)}) = 0\), es decir que hay un extremo en
\({\bf{x}}^{(0)}\), se obtiene el siguiente sistema de ecuaciones:

\[{\bf{x}}^{(i+1)} = {\bf{x}}^{(i)} - {\bf{H}^{-1}}\nabla f ({\bf{x}}^{(i)})\]

que se puede resolver de forma iterativa para obtener los valores de
\({\bf{x}}\) que corresponde a un extremo. Este método puede aplicar
para encontrar el mínimo de la función \(\chi^2\) a partir de las
derivadas primeras y segundas de la misma con respecto a los parámetros
\(a_j\).

Una vez que se han estimado los valores óptimos para los parámetros
\(\hat{a}_j\), la matríz de covarianza \(V\) puede obtenerse utlizando
la matríz de información de Fisher, que se puede relacionar con matríz
hessiana para los mejores valores de los parámetros obtenidos. La
inversa de la matriz de covarianza para el \(\chi^2\) se obtiene como:

\[(V^{-1})_{ij} = \left. \frac{1}{2} \frac{\partial^2 \chi^2 }{\partial a_i \partial a_j } \right|_{a_j = \hat{a}_j}\]

en donde el error de los parámetros estimados se puede obtener como la
raíz cuadrada de los elementos de la diagonal de la matríz de covarianza
\(V\).

Adicionalmente, en este trabajo se va a estimar el sesgo y el error en
los parámetros utilizando simulaciones Monte Carlo. En las simulaciones
se va suponer que los errores son gaussianos y se van a realizar un
total de \(N\) simulaciones. Para cada simulación \(s\), el valor de
\(y_i^s\) puede obtenerse como:

\[y_i^s = f(x_i;a_j) + n_i^s ~{\rm{con}}~ s=1,...,N\]

en donde \(n_i^s\) es el error simulado para cada punto de datos que es
un número aleatorio que sigue una distribución gaussiana con una
desviación estandar \(\sigma_i\) y \(a_j\) son los parámetros del
modelo. El valor de los parámetros de ajuste \(\hat{a}_j\) puede volver
a ser estimado para cada simulación, a partir de los cuales se puede
estimar el sesgo \(b_j\) y el error \(\sigma({\hat{a}_j})\):

\[b_j = \left< \hat{a}_j^s \right> - \hat{a}_j^{data} \]

\[\sigma({\hat{a}_j}) = \sqrt{\left< \hat{a}_j^2 \right> - \left< \hat{a}_j \right>^2 }\]

en donde los paréntesis angulares indican el promedio sobre todas las
simulaciones, \(\hat{a}_j^2\) es la estimacion del parámetro para cada
simulación y \(\hat{a}_j^{data}\) es la estimación de los parámetros de
los datos. Si el sesgo de un estimador es cero, este estan insesgado. La
estimación final del los parámetros se puede obtener según:

\[a^{final}_i = (\hat{a}_i - b_i) \pm \sigma(\hat{a}_i)\]

que se puede comparar con la obtenida a partir de información de Fisher.


    \section{Implementación}


    Se ha implementado un programa de Fortran, que se adjunta con este
informe, para ajustar por mínimos cuadrados una función
\(y=\exp(ax)+bx^2\) a una serie de datos mediante el método de
Newton-Raphson de acuerdo con el algoritmo descrito en el apartado
anterior.

La mayor parte de la implementación se ha realizado en forma de
funciones y subroutinas en el módulo \emph{NRmin\_mod} (archivo
\emph{NRmin\_mod.f90}). Estos procedimientos son invocados desde el
programa \emph{NRmin} (archivo \emph{NRmin.f90}), en el cual se pueden
configurar algunos de los parámetros de ejecución como el nombre de
archivo de datos, la tolerancia del ajuste, los valores iniciales de a y
b o el número de simulaciones. Aunque esta forma de configuración tiene
la desventaja de que es necesario compilar el programa cada vez que se
cambian los parámetros, ahorra la necesidad de introducir los parámetros
en cada ejecución o desarrollar un programa para la lectura de un
archivo de configuración.


    \section{Compilación, Ejecución y Resultados}


    Se ha creado un archivo Makefile para automatizar la compilación de los
módulos necesarios y la creación del ejecutable, una vez que se han
configurado los parámetros deseados en el el archivo \emph{NRmin.f90},
cuya llamada se demuestra a continuación:

    \begin{Verbatim}[commandchars=\\\{\}]
{\color{incolor}In [{\color{incolor}1}]:} \PY{o}{!}\PY{n+nb}{cd} ../src; make
\end{Verbatim}

    \begin{Verbatim}[commandchars=\\\{\}]
gfortran -c -O3  mod\_chi2.f90
gfortran -c -O3  mod\_random.f90
gfortran -c -O3  NRmin\_mod.f90
gfortran -c -O3  NRmin.f90
gfortran -o NRmin.bin -O3  mod\_chi2.o mod\_random.o NRmin\_mod.o NRmin.o
    \end{Verbatim}

    Una vez compilado el programa, se ha creado el archivo binario
\emph{NRmin.bin}, cuya ejecución genera el siguiente resultado para el
ajuste a realizar:

    \begin{Verbatim}[commandchars=\\\{\}]
{\color{incolor}In [{\color{incolor}2}]:} \PY{o}{!}\PY{n+nb}{cd} ../src; ./NRmin.bin
\end{Verbatim}

    \begin{Verbatim}[commandchars=\\\{\}]
--- importing data ---
 Opening file: ../data/exp.dat     
 \# data entries          25
 --- minimization of \textbackslash{}Chi\^{}2 by Newton-Raphson ---
step    0: a =  1.00000  b =  1.00000 \textbackslash{}Chi\^{}2 =   41.448
step    1: a =  0.70055  b =  1.77710 \textbackslash{}Chi\^{}2 =   17.354
step    2: a =  0.71956  b =  1.73214 \textbackslash{}Chi\^{}2 =   17.253
step    3: a =  0.71976  b =  1.73195 \textbackslash{}Chi\^{}2 =   17.253
step    4: a =  0.71976  b =  1.73195 \textbackslash{}Chi\^{}2 =   17.253
 convergence reached!
 --- estimating covariance matrix ---
 --- NR minimization results --- 
a parameter: 0.71976 \textbackslash{}pm 0.11895
b parameter: 1.73195 \textbackslash{}pm 0.15721
 --- goodness of fit ---
Chi\^{}2 value 17.25
Degrees of freedom:  23
Probability chi\^{}2(real)>chi\^{}2(obtained): 0.797
 --- MC simulation ---
 \# simulations:       100000
a parameter -> bias = -0.00264 error= 0.11788
b parameter -> bias = -0.00640 error= 0.15758
 --- final results with simulation --- 
a parameter: 0.72240 \textbackslash{}pm 0.11788
b parameter: 1.73835 \textbackslash{}pm 0.15758
 --- end of the program ---
    \end{Verbatim}

    En este caso se han leido los datos provistos y se ha utilizado el
método de Newton-Raphson con una tolerancia de \(10^{-5}\) y unos
valores iniciales \([1.0,1,0]\) (se han comprobado otros valores
iniciales para comprobar que el no se encontraba en un mínimo relativo)
para \(a\) y \(b\). Se observa que método converge muy rapidamente, en 4
iteraciones, proporcionando unos valores de ajuste para los parámetros
de:

\[ a = 0.72 \pm 0.12 \] \[ b = 1.73 \pm 0.16 \]

las incertidumbres de los parámetros han sido en este caso calculadas a
partir de la matrix de información de Fisher. El valor mínimo para el
\(\chi^2 = 17.25\) para \(N-p=23\) grados de libertad, lo que
corresponde a una probabilidad de bondad del ajuste \(p=0.797\), que es
mucho mayor que \(0.01\) por lo que se ha obtenido un buen ajuste.

Se han realizado un total de 100000 simulaciones, de la forma descrita
en el fundamento teórico, para estimar valores precisos para el sesgo y
error que se muestran en la salida de la ejecución del programa. Los
valores obtenidos para el sesgo son pequeños (\(b_1 \approx -0.003\) y
\(b_2 \approx -0.006\)) pero no se puede considerar que el estimador
este insesgado. Los errores estimados a partir de la simulación han sido
de \(\sigma(a) \approx 0.12\) y \(\sigma(b) \approx 0.16\) que coinciden
con estimados a partir de la matríz de Fisher por lo que se comprueba la
validez de este método.

    A continuación, se presentan los datos experimentales conjuntamente con
el modelo a ajustar utilzando los parámetros obtenidos por el programa
desarrollado. La visualización provista se ha realizado en Python, con
ayuda de las bibliotecas de software NumPy y Matplotlib.

    \begin{Verbatim}[commandchars=\\\{\}]
{\color{incolor}In [{\color{incolor}3}]:} \PY{o}{\PYZpc{}}\PY{k}{matplotlib} \PY{n}{inline}
        \PY{k+kn}{import} \PY{n+nn}{numpy} \PY{k+kn}{as} \PY{n+nn}{np}
        \PY{k+kn}{import} \PY{n+nn}{matplotlib.pyplot} \PY{k+kn}{as} \PY{n+nn}{plt}
        \PY{c}{\PYZsh{} import and plot data}
        \PY{n}{dat} \PY{o}{=} \PY{n}{np}\PY{o}{.}\PY{n}{genfromtxt}\PY{p}{(}\PY{l+s}{\PYZdq{}}\PY{l+s}{../data/exp.dat}\PY{l+s}{\PYZdq{}}\PY{p}{)}
        \PY{n}{plt}\PY{o}{.}\PY{n}{figure}\PY{p}{(}\PY{n}{figsize}\PY{o}{=}\PY{p}{(}\PY{l+m+mf}{12.0}\PY{p}{,}\PY{l+m+mf}{8.0}\PY{p}{)}\PY{p}{)}
        \PY{n}{plt}\PY{o}{.}\PY{n}{errorbar}\PY{p}{(}\PY{n}{dat}\PY{p}{[}\PY{p}{:}\PY{p}{,}\PY{l+m+mi}{0}\PY{p}{]}\PY{p}{,}\PY{n}{dat}\PY{p}{[}\PY{p}{:}\PY{p}{,}\PY{l+m+mi}{1}\PY{p}{]}\PY{p}{,}\PY{n}{dat}\PY{p}{[}\PY{p}{:}\PY{p}{,}\PY{l+m+mi}{2}\PY{p}{]}\PY{p}{,} \PY{n}{fmt}\PY{o}{=}\PY{l+s}{\PYZdq{}}\PY{l+s}{.k}\PY{l+s}{\PYZdq{}}\PY{p}{)}
        \PY{n}{plt}\PY{o}{.}\PY{n}{xlabel}\PY{p}{(}\PY{l+s}{\PYZdq{}}\PY{l+s}{x}\PY{l+s}{\PYZdq{}}\PY{p}{)}
        \PY{n}{plt}\PY{o}{.}\PY{n}{ylabel}\PY{p}{(}\PY{l+s}{\PYZdq{}}\PY{l+s}{y}\PY{l+s}{\PYZdq{}}\PY{p}{)}
        \PY{c}{\PYZsh{} generate model data}
        \PY{n}{a} \PY{o}{=} \PY{l+m+mf}{0.72} \PY{c}{\PYZsh{} /pm 0.12}
        \PY{n}{b} \PY{o}{=} \PY{l+m+mf}{1.73} \PY{c}{\PYZsh{} /pm 0.16}
        \PY{n}{x} \PY{o}{=} \PY{n}{np}\PY{o}{.}\PY{n}{linspace}\PY{p}{(}\PY{n}{np}\PY{o}{.}\PY{n}{min}\PY{p}{(}\PY{n}{dat}\PY{p}{[}\PY{p}{:}\PY{p}{,}\PY{l+m+mi}{0}\PY{p}{]}\PY{p}{)}\PY{p}{,}\PY{n}{np}\PY{o}{.}\PY{n}{max}\PY{p}{(}\PY{n}{dat}\PY{p}{[}\PY{p}{:}\PY{p}{,}\PY{l+m+mi}{0}\PY{p}{]}\PY{p}{)}\PY{p}{,} \PY{l+m+mi}{1000}\PY{p}{)}
        \PY{n}{plt}\PY{o}{.}\PY{n}{plot}\PY{p}{(}\PY{n}{x}\PY{p}{,} \PY{n}{np}\PY{o}{.}\PY{n}{exp}\PY{p}{(}\PY{n}{x}\PY{o}{*}\PY{n}{a}\PY{p}{)}\PY{o}{+}\PY{n}{b}\PY{o}{*}\PY{n}{x}\PY{o}{*}\PY{o}{*}\PY{l+m+mi}{2}\PY{p}{,} \PY{l+s}{\PYZdq{}}\PY{l+s}{\PYZhy{}\PYZhy{}r}\PY{l+s}{\PYZdq{}}\PY{p}{)}
\end{Verbatim}

            \begin{Verbatim}[commandchars=\\\{\}]
{\color{outcolor}Out[{\color{outcolor}3}]:} [<matplotlib.lines.Line2D at 0x1060b4510>]
\end{Verbatim}
        
    \begin{center}
    \adjustimage{max size={0.9\linewidth}{0.9\paperheight}}{NRmin-report_files/NRmin-report_13_1.png}
    \end{center}
    { \hspace*{\fill} \\}
    
    Se observa, como se esperaba dado el alto valor de la bondad del ajuste,
que los parámetros obtenidos ajustan bien los datos experimentales a
partir del modelo propuesto.


    \section{Conclusión}


    Se ha implementado existosamente un programa en Fortran para el ajuste
de una función no lineal a unos datos experimentales mediante
minimización del \(\chi^2\) por método de Newton-Raphson. Las
incertidumbres de los parámetros han sido obtenidas tanto a partir de la
matríz de información de Fisher como de simulaciónes Monte Carlo.

Se ha obtenido un valor óptimo de los parámetros para los datos
provistos de \(a = 0.72 \pm 0.12\) y \(b = 1.73 \pm 0.16\). Los errores
obtenidos mediante simulaciones coinciden con los obtenidos a partir de
la matríz de Fisher.


    % Add a bibliography block to the postdoc
    
    
    
    \end{document}
